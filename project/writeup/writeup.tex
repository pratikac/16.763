\documentclass[letterpaper, 10pt, twocolumn, reqno]{amsart}
%\documentclass[letterpaper, 11pt, reqno]{amsart}

\usepackage[margin=1in]{geometry}
\usepackage{mathpazo}
% packages
\usepackage{amsmath, amsfonts, amssymb, bm, enumerate, url} %, flushend}
\usepackage[boxruled, vlined, linesnumbered]{algorithm2e}
\usepackage[usenames, dvipsnames]{color}
\usepackage[pdftex, xetex]{graphicx}
\usepackage[font={small}]{caption}
\usepackage{float, colortbl, tabularx, multirow, subcaption, environ, wrapfig}
\usepackage{pgf, tikz}
\usetikzlibrary{arrows,automata}
\usepackage[normalem]{ulem}

\usepackage[bookmarks=true]{hyperref}
%\usepackage[hyperpageref]{backref}
\hypersetup{
colorlinks=true, linkcolor=red, citecolor=blue, filecolor=magenta, urlcolor=blue
%linkcolor=black, citecolor=black, filecolor=black, urlcolor=black
}

\usepackage{url, cite}

% macros
\definecolor{darkgreen}{rgb}{0,0.6,0} \newcommand{\dg}{\color{darkgreen}}
\definecolor{fullred}{rgb}{0.85,.0,.1} \newcommand{\fr}{\color{fullred}}
\definecolor{brown}{rgb}{0.65,0.16,0.16} \newcommand{\br}{\color{brown}}
\definecolor{orange}{rgb}{1,0.5,0} \newcommand{\oran}{\color{orange}}
\newcommand{\bl}{\color{blue}}
\newcommand{\pcmargin}[2]{{\dg #1}\marginpar{\tiny\noindent{\raggedright{\oran[PC]}\bl{ #2} \par}}}
\newcommand{\pc}[2]{\pcmargin{#1}{#2}}


%\newcommand{\myclearpage}{\clearpage}
\newcommand{\myclearpage}{}

% math macros
\newcommand{\abs}[1]{\ensuremath \left| #1 \right|}
\newcommand{\norm}[1]{\ensuremath \lVert#1\rVert}
\newcommand{\given}{\, \vert \,}
\renewcommand{\cal}[1]{\ensuremath \mathcal{#1}}
\newcommand{\qed}{\hfill \mbox{\raggedright \rule{0.1in}{0.1in} } }

\newcommand{\aeq}[1]{\begin{align} #1 \end{align}}
\newcommand{\aeqs}[1]{\begin{align*} #1 \end{align*}}
\newcommand{\beq}[1]{\begin{equation}#1\end{equation}}
\newcommand{\beqs}[1]{\begin{equation*}#1\end{equation*}}
\newcommand{\trm}[1]{\ensuremath \textrm{#1}}
\newcommand{\enum}[2]{\begin{enumerate}[#1]{#2}\end{enumerate}}
\newcommand{\ilist}[1]{\begin{itemize}{#1}\end{itemize}}
\newcommand{\ag}[1]{\ensuremath \left\langle#1\right\rangle}
\newcommand{\bee}{\begin{enumerate}}
\newcommand{\eee}{\end{enumerate}}
\newcommand{\bmat}[1]{\begin{bmatrix}#1\end{bmatrix}}
\newcommand{\mpage}[2]{\begin{minipage}{#1}#2\end{minipage}}
\newcommand{\la}{\leftarrow}
\newcommand{\ra}{\rightarrow}

\usepackage{theorem}
\newtheorem{theorem}{Theorem}
\newtheorem{proposition}[theorem]{Proposition}
\newtheorem{lemma}[theorem]{Lemma}
\newtheorem{corollary}[theorem]{Corollary}
\newtheorem{problem}[theorem]{Problem}
%\theoremstyle{definition}
\newtheorem{definition}[theorem]{Definition}
\newtheorem{example}[theorem]{Example}
\newtheorem{note}[theorem]{Note}
\newtheorem{remark}[theorem]{Remark}
%\theoremstyle{plain}
\newtheorem{assumption}[theorem]{Assumption}

\usepackage[marginal]{showlabels}
\renewcommand{\showlabelfont}{\footnotesize\slshape\color{brown}}

\linespread{1.05}

% \newcommand{\algsize}{\footnotesize}
% \setlength{\floatsep}{0.0in}
% \setlength{\textfloatsep}{0.0in}
% \setlength{\intextsep}{0.0in}
% \setlength{\belowcaptionskip}{0.05in}
% \setlength{\abovecaptionskip}{0.1in}
% \setlength{\abovedisplayskip}{0.05in}
% \setlength{\belowdisplayskip}{0.05in}

\graphicspath{{../fig/}}

\title{Column Generation Techniques in Air Transportation}
\author{Pratik Chaudhari$^*$}
\thanks{$^*$ Laboratory of Information and Decision Systems, MIT.\newline
Email: \href{pratik.ac@gmail.com}{pratik.ac@gmail.com}}
\date{May 10, 2014}

\begin{document}
\begin{abstract}
This project
\end{abstract}
\maketitle

\section{Introduction}
\label{sec:intro}

Typically in air transportation problems such as crew scheduling and fleet assignment, there are constraints on sequences of flights using a resource, e.g., an aircraft or a crew member. Although the number of constraints is huge, the problem can be made tractable by enumerating all possible sequences and assigning decision variables to each one of them. Column generation~\cite{desrosiers2005primer} then is a formalization of the simple idea that constraints on these sequences can be eliminated by only considering sequences that satisfy these constraints.

More formally, if the entire set of variables (columns) is unnecessary for the solution, the basis variables (cf.\@ simplex algorithm) can be generated as needed. This is akin to solving a \emph{restricted master problem} in conjunction with smaller sub-problems, also known as \emph{pricing problems}. This sub-problem, roughly speaking, finds the new basis variable by minimizing the reduced cost of each new variable using the dual variables in the restricted problem. We iterate to construct a new master problem by adding this new variable to the optimization. The efficacy of column generation (also known as \emph{branch-and-price}~\cite{barnhart1998branch} for integer programs) hinges on an efficient solution to the pricing problem. For example, in fleet assignment, maintenance routing and crew scheduling, the master problem is a \emph{set-cover problem} while the pricing problem is form of \emph{shortest path problem} which can be easily solved using dynamic programming. Various similar problems like constrained or multi-label shortest path problems are viable candidates for sub-problems in branch-and-price methods. After giving a brief overview of the algorithm, we will demonstrate an implementation of column generation techniques on canonical problems such as \emph{resource-constrained shortest path} and \emph{cutting-stock problem}. The key idea here is to express, e.g., the shortest path problem, not as a combination of individual arcs, but instead as a decision variable for whether a particular sub-path is included in the shortest path or not.

Now consider the problem of \emph{aircraft scheduling}, which we define to be the combination of four sub-problems, viz., \emph{schedule design, fleet assignment, maintenance routing} and \emph{crew scheduling}. The sheer size of each of these individual problems makes it almost impossible to solve the combined problem of aircraft scheduling, even for a small airline. Practically, there are huge advantages to be gained in terms of optimality or even robustness if we can solve the combined problem. Due to the large number of constraints, it is a promising domain for techniques like branch-and-price. In particular, we will focus on integrating aircraft maintenance and crew scheduling. As suggested in~\cite{cohn2003improving}, elements of the two problems can be integrated to ensure that only the maintenance routings that relevant to the crew scheduling problem are considered. The key idea in this approach is the notion of \emph{short connect}, which are routings that can be realized only if the crew stays on the aircraft. Evidently, a set of short connects can represent numerous different routings, which is crucial for a quick solution to the problem. We will present results of computational experiments using column generation on formulations that integrate the two problems. These formulations typically involve solving an integer linear program using branch-and-bound techniques, while the LP relaxation is solved using column generation; the pricing problem in this case will be a multi-label shortest path problem.

\section{Column Generation}
\label{sec:column_generation}

This section introduces column generation for solving large linear programs using the cutting-stock problem and resource-constrainted shortest path as examples. We also describe the simplex algorithm and the dual problem formulation, this will be help us to motivate as well as introduce notation for column generation.

\subsection{Linear programs: Standard form}
\label{ssec:lp_standard}

The material here is presented from~\cite{bertsimas1997introduction}. We will assume that the linear program is given to us in the standard form as follows:
\aeq{
    \trm{minimize}\ &c' x, \notag \\
    \trm{subject to}\ & Ax = b, \notag \\
    & x \geq 0.
    \label{eqn:lp_standard_form}
}
where $c \in \reals^n$ is the cost-vector, $A = [a_{ij}] \in \reals^{m \times n}$ is the constraint matrix with $m$ constraints and $b \in \reals^m$. The columns of $A$, which we denote by $A_1, \ldots, A_n$ are also known as the resource vectors while $b$ is the target vector; motivated from
$$\sum_{i=1}^n A_i x_i =b.$$
So in the standard form, we wish to synthesize $b$ using a non-negative amount, $x_i$, of each resource $A_i$. Note that a different form of linear program, e.g., with inequality constraints can be easily converted to the standard form as follows,
\clist{
    \item if $x_i$ is unrestricted, write it as $x_i = x_i^+ - x_i^-$ where both $x_i^+, x_i^- \geq 0$;
    \item construct slack variables for inequality constraints, i.e., if $\sum_{j=1}^n a_{ij} x_j \leq b_i$, we use
    $$\sum_{j=1}^n a_{ij} x_j + s_i = b_i, \quad s_i \geq 0;$$
    \item finally, note that a maximization is just another minimization problem with a negative cost vector $c$.
}
Let $P = \cbrac{x \given Ax \geq b, -Ax \geq -b, x \geq 0}$ be a polyhedron, note that $P$ is the space of feasible candidates for Prob.~\eqref{eqn:lp_standard_form}.

\subsection{Simplex algorithm}
\label{ssec:simplex}
The simplex algorithm is based upon the observation that the optimal solution always lies on the corners of $P$. Note that since $b \in \reals^m$, if the feasible solution $x$ were of length $m$, we can just invert the corresponding rows and obtain $x$. To that end, if $x$ is a \emph{basic feasible solution}, let $B_I = \cbrac{B_1, \ldots, B_m}$ be the indices of the basic variables, i.e, $x_j = 0$ for all other indices. Let $B = \bmat{A_{B_1}, \ldots, A_{B_m}}$ be the corresponding basis matrix. we thus have,
$$
x_B = B^{-1} b
$$
where $x_B = \bmat{x_{B_1}, \ldots, x_{B_m}}$.
Let us consider moving away from $x$ to a new vector $x + \th d$ by selecting some new variable $x_j$ and making it non-zero, i.e, $d_j = 1$ for some $j \notin B_I$. Since $A(x+\th d) = b$, we need $A d = 0$ which implies
$$
0 = A d = \sum_{i=1}^n A_i d_i = \sum_{i=1}^m A_{B_i} d_{B_i} + A_j = B d_B + A_j.
$$
Since $B$ is invertible, we get $d_B = -B^{-1} A_j$. Note that for a small $\th$, the feasiblity is still maintained as we move away from $x_B$ towards $x_B + \th d$. Also, the rate of cost change by this move is
\beq{
\bar{c}_j = c_j + c_B' d_B = c_j - c_B' B^{-1} A_j
\label{eqn:reduced_cost}
}
which is known as the ``reduced cost'' of variable $x_j$. In other words, $\bar{c}_j$ is the change in cost incurred after a unit increase in the variable $x_j$. Of course, if this is positive for all variables in the problem, we know that we have found the solution.

The simplex algorithm then starts with a basic feasible solution and iteravtively adds new variables to this set. While doing so, we would like to pick a $\th$ s.t. $x+\th d$ is a large step, i.e.,
$$
\th^* = \max \cbrac{\th \given x + \th d \in P}.
$$
It can be shown that $\th^*$ has a simple formula
$$
\th^* = \min_{i \in [m],\ d_{B_i} < 0}\ \rbrac{- \f{x_{B_i}}{d_{B_i}}}.
$$
Thus the iteration of the simplex algorithm are simple.
\enum{
    \item Start with $A_B$ and  current basic solution $x$, compute the reduced costs $\bar{c}_j = c_j - c_B' B^{-1} A_j$ for all non-basic variables $x_j$; if all
    $\bar{c}_j \geq 0$, $x$ is optimal solution, else pick some $j$ with $\bar{c}_j < 0$.
    \item Calculate $\th^* = \min_{i \in [m],\ d_{B_i} < 0}\ \rbrac{- \f{x_{B_i}}{d_{B_i}} }$. If there is no $i$ s.t. $d_{B_i} < 0$, the optimal cost is $-\infty$.
    \item Let $l$ be the minimizing index in step 2., form a new basis by replacing $A_{B_l}$ with $A_j$ and set ${x_j \la \th^*}$ and $x_{B_i} \la x_{B_i} - \th^* d_{B_i}$.
 }
The above algorithm can be made much more efficient by passing information about dual variables in consecutive steps.

\subsection{Dual problem}
\label{ssec:dual}

We can write the Prob.~\eqref{eqn:lp_standard_form} as
\aeqs{
    \trm{minimize}\ & c'x + p'(b - Ax), \\
    \trm{subject to}\ & x \geq 0.
}
where $p \in \reals^m$ is the Lagrange multiplier. This new problem has fewer constraints and hence by definition,
\aeqs{
g(p) = \min_{x \geq 0} \sqbrac{c'x + p'(b-Ax)} &\leq c'x^* + p'(b-Ax^*) \\
&=c' x^*.
}
Here $x^*$ is the optimal solution of Prob.~\eqref{eqn:lp_standard_form} (also known as the ``primal'') and the last equality is true because $p$ at optimality is such that all constraints are satisfied. The new problem,
\aeqs{
    \trm{maximize}\ &g(p), \\
    \trm{subject to}\ &\trm{no constraints}
}
thus searches for the best lower bound of the primal. Note that
\aeqs{
g(p) = \begin{cases}
p' b \quad & \trm{if}\ c' \geq p'A\\
-\infty, \quad & \trm{otherwise}.
\end{cases}
}
Thus, the ``dual'' problem of Prob.~\eqref{eqn:lp_standard_form} is
\aeq{
    \trm{maximize}\ \quad& p' b, \notag\\
    \trm{subject to}\ \quad& p'A \leq c'.
}
Dual problems are important for many reasons, viz., (i) the dual of a dual is the primal, (ii) dual solution is a lower bound for the primal's cost and finally, (iii) if the primal has a solution, so does the dual, and moreover, their costs are the same.

\subsection{Cutting-stock problem}
\label{ssec:col_gen_cutting_stock}

Consider a paper mill with a number of rolls of paper of fixed width available to it. Different customers want different numbers of rolls of various-sized widths. The problem is to them efficiently cut ``patterns'' in each roll of paper so as to avoid waste. To get an idea of the magnitude of the problem, note that European paper mills produced a total turnover of almost $\$ 75$ million in 2012 and hence saving even 1\% of this is a significant advantage.

Let the width of each roll be $W$ and $m$ customers want $n_i$ rolls each of width $w_i$. If $K$ is the number of rolls available, let $y_k = 1$ if a roll $k \in [K]$ is cut and zero otherwise. Let $x_i^k$ be the number of times item $i$ is cut on roll $k$. We then have the following formulation.
\aeq{
    (P_1) \quad \min &\sum_{k \in [K]} y_k, \notag \\
    \trm{s.t.}\ & \sum_{k \in [K]} x_i^k \geq n_i \quad \forall\ i \in [m], \qquad \trm{(demand)}\notag\\
    & \sum_{i=1}^n w_i x_i^k \leq W y_k \quad \forall k \in [K],\quad \trm{(width of roll)} \notag\\
    &x_i^k \in \integers_+, y_k \in \cbrac{0,1}.
    \label{eqn:cutting_stock_ip}
}
Note that the above problem has $Kn + K$ variables and $(n+K)$ constraints and since it is an integer program, solving it for even small values of $K, n$ is expensive. The real reason for this is that the LP relaxation of this problem is bad. Note that
$$
\sum_k y_k \geq \f{1}{W} \sum_k \sum_i w_i\ x_i^k \geq \f{1}{W} \sum_i n_i\ w_i
$$
which is a naive bound and says that the number of rolls needed is simply the sum of all demands divided by $W$. Let use pose the problem in a different way as follows.

We call the set of distinct cuts of a roll as a pattern, i.e., if $W=100$, and we make 4 cuts of $w_1 = 25$ each, we generate a patterm $\{ w_1, w_1, w_1, w_1 \}$. If on the other hand, we make one cut of $w_1 =25$ and two of $w_2 = 35$, we generate a pattern $\{ w_1, w_2, w_2 \}$. Let $x_j$ be the number of times a pattern $j$ is used. Let $a_{ij}$ be the number of times item $i$ is cut in pattern $j$. Consider then, the new problem --
\aeq{
    (P_2) \quad \min &\sum_{j=1}^n x_j, \notag \\
    \trm{s.t.}\ & \sum_{j=1}^n a_{ij} x_j \geq n_i \quad \forall\ i \in [m], \qquad \trm{(demand)} \notag\\
    &x_j \in \integers_+, j \in [n].
    \label{eqn:cutting_stock_pattern_ip}
}
where $n$ is the number of valid patterns, i.e., ones that satisfy $\sum_{i=1}^m w_i a_{ij} \leq W$. Note that the new problem is also an IP and moreover, it contains a huge number of variables, almost $m \choose k$, where $k$ is the average number of items in a pattern. However, let's look at its LP relaxation. We replace the integrality constraints with
$$
x_j \in \reals_+,\quad j \in [n].
$$
We call the new problem, the ``Linear Programming Master (LPM)''; the solution to this could very well be non-integral, since we have $\geq$ in the constraints we can generate a feasible, integral solution by rounding up. It has a large number of variables but only $m$ constraints. Use now use the simplex algorithm from Sec.~\ref{ssec:simplex} to solve this LP.

The main idea of column-generation is that most of variables are zero, i.e., they do not form the basis (since the basis has only as many variables as constraints). Hence, we only need a small subset of all the variables to form the solution. How do we leverage this observation? Let us start with a subset $\cP \subset [n]$ of columns of LPM and call it the ``Restricted Linear Programming Master (RLPM)''and now look at its dual problem.
\aeq{
    \trm{maximize}\ &\sum_{i=1}^m n_i \pi_i, \notag \\
    \trm{subject to}\ & \sum_{i=1}^m \a_{ij} \pi_i \leq 1, \quad j \in \cP, \notag\\
    &\pi \geq 0, \quad i \in [m].
    \label{eqn:cutting_stock_dual_rlpm}
}
Let $\hat{\pi}$ be the optimal solution to the dual problem. The reduced cost of column $j$ in that case is given by simply
$$
1  - \sum_{i=1}^m a_{ij}\ \hat{\pi}_i.
$$
The variable with the best reduced cost, i.e., which we absolutely must include in the primal is to find
$$
\min \cbrac{1  - \sum_{i=1}^m a_{ij}\ \hat{\pi}_i \given j \in [n] \setminus \cP}
$$
which is still expensive to do since we need to enumerate all the possible patterns. However, it is another optimization problem, and a very famous at that. This problem, which we will call as the ``sub-problem'' (or the ``pricing problem'') is a Knapsack problem, i.e.,
\aeq{
    \max \quad &\sum_{i=1}^m \hat{\pi}_i y_i, \notag \\
    \trm{s.t.}\ \quad & \sum_{i=1}^m w_i\ y_i \leq W, \notag\\
    & y_i \in \integers_+, i \in [m].
    \label{eqn:cutting_stock_sub_problem}
}
Knapsack is one of the ``easy'' NP-hard problems and can be solved in $\bigo(mW)$ time by dynamic programming. We thus have a strategy for solving the cutting-stock problem as follows.
\enum{
    \item Start with some initial columns of LPM, e.g., use a simple pattern to cut rolls into $\ceil{W/w_i}$ pieces of $w_i$. $A = [a_{ij}]$ in Prob.~\eqref{eqn:cutting_stock_pattern_ip} is thus diagonal.
    \item repeat until all reduced costs are non-negative
        \enum{
            \item Solve RLPM and let $\hat{\pi}$ be the dual variables
            \item Find a new column as the solution of knapsack pricing problem with the most negative reduced cost
            \item Add new column to RLPM
        }
}

Note that this procedure only yields a solution to LPM, i.e., the relaxed problem of Prob.~\eqref{eqn:cutting_stock_pattern_ip}. In order to get an integer solution, observe that
$$
\sum_{j=1}^n a_{ij} \ceil{x_j} \geq \sum_{j=1}^n a_{ij} x_j \geq n_i
$$
which means that the rounded-up solution is also a feasible solution for Prob.~\eqref{eqn:cutting_stock_pattern_ip}. But of course, this rounded-up solution is not optimal, to obtain the optimal solution to the original IP, we have to take recourse to ``branch and bound'', which is a technique for solving general integer linear programs. In the next section, we discuss this procedure and motivate the ``branch and price'' framework which uses column generation in addition to branch and bound.

\section{Branch and Price}
\label{sec:branch_and_price}

\subsection{Branch and bound}
\label{ssec:branch_and_bound}

Let us construct a simple procedure to solve general MIPs. We first construct
the relaxation and solve it iteratively by branching off non-integral solutions to successive relaxations. The idea is that the solution of a
relaxed LP, by definition, is a lower bound on the optimal cost of the MIP. Say $\hat{x}$ be the current cost. If the solution is integral, we are done. If not, we pick any non-integral variable in the solution, say $\hat{x}_i = 4.2$ and create two branches $\hat{x}_i \leq 4$ and $\hat{x}_i \geq 5$. The tree
is explored in a breadth-first manner until we get a fully integral solution. We can also prune certain branches by noting that if the solution of the
relaxed LP at some node is less than the current minimum solution, we do not need to explore that branch downwards. We use a priority queue to sort the nodes of the branch-and-bound algorithm by their LP relaxation lower bounds and pop from this queue at every iteration. Typically, one might also construct an upper bound to guide the search. The final search procedure is as shown in Alg.~\ref{alg:bnb}.

\IncMargin{0.04in}
\begin{algorithm}[!h]
\footnotesize
$J_l$: LP relaxation of original problem\;
$J_u$: upper bound with integer solution (possibly heuristic)\;
$J^* \la J_u$: incumbent best solution\;
\vspace{0.05in}
\While{nodes in tree}
{
    \enum{
    \item solve current LP, compute duals
    \item update lower bound $J_l$
    \item If solution $J_c$ is fractional,
        \begin{enumerate}[(a)]
        \item if $J_c \geq J_u$, discard node
        \item if $J_c < J_l$, create two new nodes (branch)
        \end{enumerate}
    \item If integral solution,
        \begin{enumerate}[(a)]
        \item if $J_c < J_u$, update upper bound
        \item discard node
        \end{enumerate}
    }
}
\caption{Branch and bound}
\label{alg:bnb}
\end{algorithm}
\DecMargin{0.04in}

% \begin{figure}
% \centering
% \includegraphics[width=0.8 \columnwidth]{bnb}
% \caption{Branch and bound search tree}
% \label{fig:bnb}
% \end{figure}

After creating a flight schedule, an airline must decide the capacity/ aircraft type for each flight, i.e., fleet assignment. It then allocates sequences of flights to a particular aircraft so as to maximize revenues and avoid aircrafys flying back to the hub every time; this problem is known as fleet assignment. The next step in this process is to ensure that every tail number, i.e., physical aircraft regularly visits a maintenance base station, this problem is called as maintenance routing. The last step in the aircraft scheduling process is to allocate crew to each flight, after incorporating various constraints, viz., flight hours, home bases etc., this problem is known as crew pairing.

In this section, we focus on the last two problems. We first discuss each problem along with its mathematical formulation. Potentially, integrating some parts of the aircraft scheduling process can provide large benefits in terms of optimality and cost. Towards this end, we look at integrating maintenance routing and crew pairing together. We first describe a naive integration which couples all the constraints together and use this to motivate a more efficient solution which leverages a concept known as ``short connects''.

\subsection{Crew pairing}
\label{ssec:crew_pairing}

The crew pairing problem is a classical set partitioning problem, we would like to minimize the cost of all crew pairings subjects to the constraint that every flight has a crew. Let
\clist{
    \item $P$ be the set of feasible pairings;
    \item $F$ is the set of flights;
    \item $c_p$ is the cost of pairing $p \in P$;
    \item $\d_{fp} = \ind(\trm{flight}\ f \in\ \trm{pairing}\ p)$ is an indicator variable which is 1 if flight $f$ is included in pairing $p$;
    \item $y_p$ is a binary variable which is 1 if pairing $p$ is included in the solution.
}
The crew pairing problem can now be written as
\aeq{
    (CP) \qquad
    \min\ &\sum_{p \in P}\ c_p\ y_p, \label{eqn:cp_cost}\\
    \trm{s.t.}\ & \sum_{p \in P}\ \d_{fp} y_p = 1 \quad \forall\ f \in F, \label{eqn:cp_cover}\\
    & y_p \in \cbrac{0, 1} \quad \forall\ p \in P.
     \label{eqn:cp}
}

Note that \eqref{eqn:cp_cost} minimizes the cost of pairings, while constraints in \eqref{eqn:cp_cover}, also known as cover constraints ensure that each flight gets a crew. This formulation implicitly accounts for ``infeasibility rules'', i.e., the set $P$ is the set of all feasible pairings.
The crew pairing problem is relatively easy to
model mathematically and interpret as an optimization problem. Still, the task of solving the model efficiently and producing a practical crew schedule can be very challenging. However, note that this problem has $\abs{P}$ variables, which is typically very large, it is all feasible pairings in a flight network and can be of the order of hundreds of millions. Billions of legal pairings can be generated for as few as 1000 flight legs. This is exacerbated even further by hub-n-spoke network topology of most big airlines. Refer to~\cite{} for state-of-the-art approaches to this problem.

\subsection{Column generation for crew pairing}
\label{ssec:col_gen_cp}
write material from~\cite{gopalakrishnan2005airline}
multi-label shortest path


\subsection{Maintenance routing}
\label{ssec:maintenance_routing}

Similar to crew pairing, we use a ``string-based'' approach to formulating maintenance routing problems. Let us define a few additional variables,
\ilist{
    \item $R$ is the set of feasible route strings;
    \item $c_r$ is the cost of each route string $r$;
    \item $\a_{fr} = \ind \rbrac{\trm{route}\ r\ \trm{contains flight}\ f}$ is an indicator variable that is 1 if route $r$ contains flight $f$;
    \item $d_r$ is 1 if route $r$ is contained in the solution;
    \item $N$ is the set of nodes in a airline network, i.e., the set of origins and destinations of all flights in the network (indexed by time);
    \item $g_n^-, g_n^+$ are called ground arc variables, if $n = (s,t)$ is a node at a physical location $s$ and time instant $t$, they represent the number of aircraft at $s$ immediately prior to and immediately after time $t$;
    \item $R^T$ is the number of routes that span an arbitrary time $T$, also known as the countline;
    \item $N^T$ is the number of nodes with corresponding ground arcs $g_n^+$ that span the countline;
    \item $K$ is the total number of aircrafts available.
}

We are now ready to formulate the maintenance routing problem as follows:
\aeq{
    (MR) \qquad
    \min\ &\sum_{r \in R} c_r\ d_r, \label{eqn:mr_cost}\\
    \trm{s.t.} \qquad \qquad&\\
    \sum_{r} \a_{fr} d_r &= 1 \quad \forall\ f \in F, \label{eqn:mr_coverage}\\
    \sum_{r\ \trm{ends in}\ n} d_r + g_n^- &= \sum_{r\ \trm{starts at}\ n} d_r + g_n^+ \quad \forall n \in N, \label{eqn:mr_flow}\\
    \sum_{r \in R^T}\ d_r + \sum_{n \in N^T}\ g_N^+ &\leq K, \label{eqn:mr_total_aircrafts}\\
    d_r &\in \cbrac{0,1} \quad r \in R,\\
    g_n^-,\ g_n^+ &\geq 0 \quad n \in N.
    \label{eqn:mr}
}

The cost function in ~\eqref{eqn:mr_cost} minimizes the cost of all routes in the solution while coverage constraints in \eqref{eqn:mr_coverage} ensure that
all flights are covered. \eqref{eqn:mr_flow} are flow constraints to ensure that the number of aircrafts that are available at a node $n$, i.e., the sum of all aircrafts on the ground and the ones that just landed is equal to the
number of aircrafts that remain on the ground and the ones that depart. Similarly, \eqref{eqn:mr_total_aircrafts} is a constraint that upper bounds
the total number of aircraft to $K$. Note that the airline network is acyclic and the flow forms a circulation, i.e., we only need ensure \eqref{eqn:mr_total_aircrafts} at some point within the planning horizon, say point $T$. Again, the number of variables in this problem is exponentially large, because the number of routes is the set of all subsequences in the airline network.

\section{Integrated model}
\label{sec:integrated_model}
Problems formulated in Sec.~\ref{ssec:crew_pairing} and Sec.~\ref{ssec:maintenance_routing}, if solved independently clearly lead to a sub-optimal solution, i.e., solving for crew pairing after solving the maintenance routing problem is sub-optimal because we are only considering a restricted set of route strings. Let us integrate the two problems, our objective to incorporate as many feasible pairings as possible using short-connects in the network, i.e., \pc{solutions when the crew and the aircraft stay together.}{}

Define a few extra variables in this context:
\ilist{
    \item $C$ is the set of short connects;
    \item $\n_{cr}$ is 1 if route $r$ contains a short connect $c$ and zero otherwise;
    \item $\eta_{cp}$ is 1 if pairing $p$ contains a short connect $c$ and zero otherwise.
}
The basic integrated probem (BIM) is written as
\aeq{
    \min\ \sum_{p \in P}\ c_p y_p,& \\
    \trm{s.t.} \qquad \qquad & \\
    \sum_{p \in P}\ \d_{fp} y_p &= 1 \quad \forall\ f \in F, \\
    \sum_{r} \a_{fr} d_r &= 1 \quad \forall\ f \in F, \\
    \sum_{r\ \trm{ends in}\ n} d_r + g_n^- &= \sum_{r\ \trm{starts at}\ n} d_r + g_n^+ \quad \forall n \in N,\\
    \sum_{r \in R^T}\ d_r + \sum_{n \in N^T}\ g_N^+ &\leq K,\\
    \sum_{r \in R}\ \nu_{cr} d_r &\geq \sum_{p \in P}\ \eta_{cp} y_p \quad \forall\ c \in C, \label{eqn:naive_short_connect}\\
    d_r &\in \cbrac{0,1} \quad r \in R,\\
    g_n^-,\ g_n^+ &\geq 0 \quad n \in N,\\
    y_p &\in \cbrac{0, 1} \quad \forall\ p \in P.
}

The only differece between BIM as compared to CP and MR is constraint ~\eqref{eqn:naive_short_connect} which says that we cannot pick a pairing containing a short connect if we do not pick a maintenance routing solution with that same short connect. A few things about this model are easy to see, (i) it has a large number of variables, the number of variables is $R + N+P$, which grows exponentially in the size of the network, (ii) \pc{the LP relaxation of this problem is a very poor lower bound.}{explain}

However, we can draw a key observation from this formulation, if we can remove the maintenance routing constraints and only consider complete solutions to the MR problem, we can potentially include CP constraints for this \emph{particular} solution as follows:
\aeq{
    \min\ \sum_{p \in P}\ c_p y_p, & \\
    \trm{s.t.} \qquad \qquad & \\
    \sum_{p \in P}\ \d_{fp} y_p &= 1 \quad \forall\ f \in F, \label{eqn:ecp_convexity}\\
    \sum_{s \in S}\ \b_{cs} x_s &\geq \sum_{p \in P}\ \eta_{cp} y_p \quad \forall\ c \in C, \label{eqn:ecp_short_connect}\\
    \sum_{s \in S}\ x_s &= 1, \\
    x_s &\in \cbrac{0, 1} \quad s \in S,\\
    y_p &\in \cbrac{0, 1} \quad \forall\ p \in P.
}
The above model, which is known as the extended crew pairing model uses the set of feasible maintenance routing solutions and connects them to crew pairing constraints using short connects. In the above problem,
\ilist{
    \item $S$ is the set of feasible maintenence routing solutions;
    \item $\b_{cs}$ is 1 if short connect $c$ is included in $s$ and zero otherwise;
    \item $x_s$ is 1 if $s$ is a part of the solution and 0 otherwise.
}
However, the number of fesible maintenance routing solutions is huge, and hence we have only just re-written the BIM problem in a different form.

\subsection{Unique maximal short connects}
\label{ssec:unique_maximal_sc}

Column generation is a promising technique to solve ECP to optimality, however the number of maintenance feasible solutions is exponential in the number of flights and hence this approach is prohibitive. Also note that only key decisions in the MR problem affect the constraints in the CP problem, viz., we can look at maintenance routing, not as a string by itself, but as a short connect for which a corresponding maintenance routing solution can be found. This observation is crucial to significantly decrease the number of variables in the optimization. Roughly, instead of constructing the set $S$ with all feasible maintenance routing solutions, we will only consider \emph{maximal unique solutions} as shown below.

\textbf{\emph{Uniqueness:}} Consider 6 flights, $a, b, \ldots, f$ and two maintenance routing (MR) solutions, viz., $a-b-c$ and $d-e-f$ or $a-b-f$ and $d-e-c$. If $a-b$ is the only short connect for these two solutions, we need to include only one of them in the set $S$, i.e., the constraints of the crew pairing problem depend only upon $a-b$ and not on the explicit routing string. This constraint is called uniqueness. Instead of a variable for every MR solution, we only need one column for every unique MR short connect. \cite{cohn2003improving} report that for an example airline's network, 8,700 distinct MR solutions could be collapsed into a single ECP column. Considering that the complexity of solving MIPs is exponentialy, this is a tremendous saving.

\textbf{\emph{Maximal solutions:}} Similarly, if we have one MR solution with SCs as $a-b$, $c-d$ and $e-f$ and another one with SCs as $a-b$ and $c-d$, the first solution is a ``maximal'' with respect to the second. In other words, any CP solution that satisfies the second will also satisfy the first. Hence, we need only consider all the ``maximal'' SCs, i.e., remove the second MR column from our problem. Again,~\cite{cohn2003improving} report that in an example network, 25,000 distinct MR solutions collapsed to just 4 maximal SCs.

We have therefore identified the kind of columns we would like to generate for ECP, they have to be unique, maximal feasible solutions of the MR problem. This is still not as useful, if we need to enumerate all the feasible MR solutions and prune the ones that do not satisfy this criteria. In the next section, we describe a procedure to efficiently construct new columns for ECP using some simple observations for the MR problem.

\subsection{Column generation}
\label{ssec:col_gen_ecp}

We would like to generate a large number of MR solutions with a large number of SCs. In order to do so, we tweak the MR problem~\eqref{eqn:mr} with a new cost function
$$
\max \sum_{r \in R}\ -c_r d_r
$$
where $c_r$ now is the number of SCs in a route string $r$. A solution to this new problem results in unique, maximal set of SCs. It however generates only a few short connects, hence we solve this problem successively using constraint
$$
\sum_{c \in C\setminus C^1}\ \sum_{r \in R}\ \nu_{cr}\ d_r \geq 1
$$
where $C^1$ is the set of short connects returned after the first iteration above. This constraint says that we would like to find an MR solution with at least one new short connect than before. We can do this iteratively, until the problem becomes infeasible, or until we arrive at a large enough number of SCs for ECP. Note it can be solved efficiently, in fact, we can use the solution of the previous iteration to bootstrap the next iteration, because it is still feasible.

\subsection{Solution procedure}
\label{ssec:ecp_solution_procedure}

It can be shown that in the ECP problem, boolean constraints on the MR variables can be relaxed~\cite{cohn2002composite}. Letting the CP variables be binary forces an integral solution for the MR variables.

We use the branch and price mechanism to solve ECP. Due to the above fact, we do not need to branch on the MR variables. This procedure works as follows ---
\enum{
    \item Create initial crew pairings and maintenance solutions
        \begin{enumerate}[(a)]
            \item We can directly use ideas from Sec.~\ref{ssec:col_gen_ecp} to generate a set of MR solutions. In fact, if we solve the MR problem~\eqref{eqn:mr}, it genrates the solution --- without adding any more MR solutions to this, we would obtain the usual sequential solution to the MR - CP airline scheduling problem.
            \item For generating initial solutions to the CP problem
        \end{enumerate}
    \item Solve current LP and compute dual variables
    \item Generate a new CP solution using reduced costs
        \begin{enumerate}[(a)]
            \item We use ideas from Sec.~\ref{ssec:col_gen_cp} to do this. However, in the ECP model, we also have to incorporate the dual variables for the short connects, i.e., the reduced cost of a pairing becomes
            $$
            c_p - \sum_{f \in F}\ \d_{fp}\ \pi_f + \sum_{c \in C}\ \eta_{cp}\ \g_c
            $$
            where $\pi_f$ is the dual variable for flight $f$ for the cover constraint~\eqref{eqn:cp_cover} and $\g_c$ is the dual variable for short connect $c$ in constraint~\eqref{eqn:ecp_short_connect}.
        \end{enumerate}
    \item Generate a new maintenance routing solution
        \begin{enumerate}[(a)]
            \item We again leverage column generation techniques here, note that the reduced cost of a maintenance column is
            $$
            0 - \sum_{c \in C} \b_{sc}\ \g_c - \s
            $$
            if $\s$ is the dual variable of the convexity constraint~\eqref{eqn:ecp_convexity} --- the contribution of MR columns to the cost function is zero. The corresponding pricing problem (cf. Sec.~\ref{eqn:cutting_stock_sub_problem}) is then
            $$
            \max \sum_{c \in C} \b_{sc}\ \g_c; \quad s \in S.
            $$
            If this solution is less than $\s$, we have a new column that we can add to the optmization. How do we solve this new problem? Note that
            $$
            -\sum_{c \in C} \b_{sc}\ \g_c = \sum_{r \in R(s)} \rbrac{-\sum_{c \in C}\ \nu_{cr}\ \g_c}
            $$
            looks very similar to the cost function~\eqref{eqn:mr_cost} with the coefficient $c_r = -\sum_{c \in C}\ \nu_{cr}\ \g_c$!

            We can also use a neat trick here, if the SCs in some MR solution $s$ are a subset of the SCs in some other solution $s'$, we would like to include $s'$ instead of $s$ as the new column. This can be accomplished by adding a small constant $\D$ to all $\g_c$.
        \end{enumerate}

    \item Solve using the branch and bound algorithm in Sec.~\ref{ssec:branch_and_bound}.

}
%Note that the LP relaxation of this problem is much better than BIM.

\section{Examples}
\label{sec:examples}

\subsection{Cutting-stock problem}
\label{ssec:eg_cutting_stock}

\subsection{Resource-constrained shortest path}
\label{ssec:eg_shortest_path}


\section{Summary}
\label{sec:summary}


{
\small
\bibliography{../writeup}
\bibliographystyle{alpha}
}
\end{document}